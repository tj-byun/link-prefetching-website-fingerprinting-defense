% csci5271.tex
% this is the main LaTex file
\documentclass{sig-alternate-05-2015}

\begin{document}

\title{Investigating the Effects of Pre-Fetching on Website Fingerprinting Attack\titlenote{This report is submitted as a partial fulfillment of {\it CSCI5271: Introduction to Security} course.}}

\numberofauthors{4}
\author{
    \alignauthor
        Vaibhav Sharma\\%
        \email{sharm361@umn.edu}
    \alignauthor
        Taejoon Byun\\
        \email{taejoon@umn.edu}
    \alignauthor
        Se Eun Oh\\%
        \email{seoh@umn.edu}
    \alignauthor
        Elaheh Ghassabani\\
        \email{ghass013@umn.edu}
    \end{tabular}\newline\begin{tabular}{c}
    \affaddr{Department of Computer Science and Engineering}  \\
        \affaddr{University of Minnesota}   \\
        \affaddr{Minneapolis, MN 55454}
}

\maketitle
\begin{abstract}
\emph{This paper is written to get an A in the CSCI5271 course. PERIOD.}

\emph{This content will be edited later.} We plan to explore the area of website fingerprinting in anonymization networks starting with the paper Website Fingerprinting in Onion Routing Based Anonymization Networks.
\end{abstract}

% We no longer use \terms command
%\terms{Theory}

\keywords{Website fingerprinting, anonymity, encrypted traffic, Tor}

\section{Introduction}
% sections/intro.tex
Anonymizing networks are privacy technologies that provide a mancinism to anonymize internet communications so as to protect users from network eavesdroppers.
Although such systems are able to hide the communication (including both routing information and content), an attacker is still able to obtain different information by analyzing the network traffic.
Network analysis can provide very rich information about message length, timing, and frequency by which an attacker can easily identify the communicating parties, and therefore bypass an anonymizing system.
This problem is known as Website Fingerprinting (WF) attack, where an adversary attempts to recognize the encrypted traffic patterns of specific web pages without using any other information \cite{juarez14, murdoch2005low}.

% we should say, there is lots of work trying to prevent WF attack. And, in this project, we studied most of the state of the art works, and tries to provide a novel defend mechanism [or something similar]

% we want to add a summary about we have done

The remaining sections of this report is organized as follows: Section~\ref{sec2} provides a brief background about Tor anonymity network and website fingerprinting and Section~\ref{sec3} mentions related works on website fingerprinting attacks, defenses and criticisms.
We suggest link prefetching as a defense mechanism in Section~\ref{sec4} followed by experimental evaluation (Section~\ref{sec5})
Section~\ref{sec6} discusses about the feasibility of the proposed defense based on experimental result, and finally conclude our report on Section~\ref{sec7}.



\section{Related Work}
% sections/related.tex
% 3. Related Works

There are a decent number of works on bypassing anonymizing systems, especially Tor, on which this section provides an overview.

\subsection{Fingerprinting Attacks}
The idea of extracting information about content of encrypted SSL packets dates back to 1996 \cite{wagner96}. Later in 2002, Hintz presented an attack on an encrypting web proxy calling it website fingerprinting. The attack successfully exploited a traffic analysis-based vulnerability to detect which website a particular user is surfing \cite{hintz2003}. Since then, a lot of research has been conducted in this area trying either to propose a more realistic attack, or a more effective defense mechanism. 

One interesting study has been done in \cite{herrmann2009}, where authors applied WF attack on different anonymizing techniques, including OpenSSH, OpenVPN, Stunnel, Tor. They tried to attack 775 different websites. The result presents a low detection rate in the closed world attack model (only a 3\% success rate for 775 pages). They used Multinomial Naive Bayes classifier focusing only on packet sizes. Later on, this attack has been improved with around 90\% accuracy for 100 wbpages in \cite{wang2013improved, cai2012touching}. In \cite{cai2012touching}, authors could defeat ad hoc defense mechanisms describing a new defense scheme that provides provable security properties. Penchenko et al. \cite{panchenko11} were also among the first to report website fingerprinting attacks with reasonable accuracy on Tor. They provided a sufficient understanding of the feature set and classification framework required for this attack. In fact, they carried out a comprehensive study of WF on Tor, which considers both closed world and open world attack models.  

Generally speaking, attacks on anonymizing networks took a variety of approaches; some of the attacks tries to discover the identity of the anonymous user, others focus on uncovering the private pathway, and others attempt to identify the servers users interact with  \cite{cai2012touching}.

\subsection{Defense Mechanisms}
Usually, defense mechanisms are developed at the IP/TCP level by changing the pattern of traffic. For example, they split packets into multiple packets, or insert fake packets into the traffic, or involve padding packets. A study performed in \cite{fu2003} shows that transmission at random intervals could be a defense against analysis-based attacks. Another defense technique proposed in \cite{wright2009} uses some tricks to change a traffic pattern in a way that it looks like another pattern. However, this method does not split or disordered packets. Therefore, attacks that work without packet size information can easily defeat this mechanism. In \cite{luo2011}, Luo et al. proposed a collection of tricks at the HTTP/TCP level as a defense mechanism against some analysis attacks. For example, they changed window sizes and order of packets in the TCP stream. Also, at the HTTP level, they tried to insert some
extra data into HTTP GET headers, generates some   

\subsection{Criticism}
% cite some papers which criticize the feasibility of the fingerprinting attack
% and provide some reasons why we think fingerprinting is still a serious
% threat that we should be prepared for.



\section{Background}
% sections/background.tex

This section provides a brief description of required background.

\subsection{Link Pre-fetching}
Today's web browsers, including Tor, makes use of a specific syntax called \emph{pre-fetching}, which was proposed as a draft standard by Mozilla. Using pre-fetching, browser can predicts documents likely to be visited by the user in the near future. Therefore, based on the hint provided by pre-fetching a browser is able to fetch those documents a head of time. In fact, it is the web page that provides a set of pre-fetching hints for the browser. Then, loading the page and passing an idle time, the browser starts to pre-fetch and cache specified documents. Needless to say, this mechanism improves efficiency. Particularly, it is most effective if the content provider may be reasonably certain which links users are going to visit next \cite{wikiPreF}.

\subsection{Network Analysis and Classifiers}
description about how we analyzed the traffic. And which classifer used, which/how features are extracted.


\section{Effects of Pre-Fetching on Fingerprinting}
In this section, we will write  about our experiments. We are planning to conduct two sets of experiments. If we consider the network traffic, the number of packages go upstream depends on the number of pre-fetching requests, and the number of downstream packages coming depends on the size of resources that should be pre-fetched. Therefore, it is obvious that pre-fetching would affect the fingerprint of the traffic of a particular website.  \emph{should be completed.}

\subsection{Investigate Pre-Fetching Effects on top 60 Popular Websites}
We are running experiments to see how pre-fetching affects the websites' fingerprints.
After doing some search on top popular websites, we put together a small crawler by which we learnt that only around 60 of all 6000 websites are use pre-fetching mechanism. We are capturing traffic of these websites in two different modes: 1) with enabled pre-fetching, and 2) with disabled pre-fetching. We are working on feature extraction, and about to decide which classifiers to use for the learning phase.
Ultimately, we plan to conduct two sets of experiments. One sort of experiment is to compare two series of the captured packets and find the accuracy number with the help of a classifier, by which our goal is to provide an evidence to see if pre-fetching really affects fingerprints of websites. So, if the result will be positive, we will perform another set of experiment, which kind of simulates a sub set of those 60 websites. Then, we will see how (altering) the size of pre-fetching affects fingerprinting attacks/ defense mechanisms.


\subsection{Effect of Pre-Fetching Packets Size on Fingerprinting Attacks}
Here, we will explain our second experiment. We will simulate a sub set of webpages we investigated in the previous experiment. Then, we will equip them with a mechanism so that they can affect the downstream traffic and finally their fingerprint. Then, we will analyze the result to see how this idea contributes to the effectiveness of attacks and defense techniques.

\section{Experiments}
% sections/experiment.tex
\begin{table}[]
\centering
\caption{Experiment design to answer {\it RQ1}}
\label{table:prefetch}
\begin{tabular}{lllll}
\cline{1-3}
\multicolumn{1}{|l|}{victim \textbackslash attacker} & \multicolumn{1}{l|}{prefetch on} & \multicolumn{1}{l|}{prefetch off} &  &  \\ \cline{1-3}
\multicolumn{1}{|l|}{prefetch on}                    & \multicolumn{1}{l|}{(1)}         & \multicolumn{1}{l|}{(2)}          &  &  \\ \cline{1-3}
\multicolumn{1}{|l|}{prefetch off}                   & \multicolumn{1}{l|}{(3)}         & \multicolumn{1}{l|}{(4)}          &  &  \\ \cline{1-3}
                                                     &                                  &                                   &  & 
\end{tabular}                  
\end{table}

\begin{enumerate}
\item
We speculate that prefetching itself might provide extra defense because of the extra packets.
It can also be the case however, prefetching websites are more vulnerable to fingerprinting because of the extra prefetch packets that shows distinct prefix.
\item
This case is unlikely since prefetching is on by default. We assume that victims will more likely be using Tor under the default setting.
\item
This case is what we are most curious about, whether a victim can confuse an attacker by simply turning prefetching setting off of his browser.
\item
This case simulates a situation where a victim is loading any other websites that does not prefetch any resource.
This can be used as a comparison case.
\end{enumerate}


\section{Conclusions}
% sections/conclusion.tex

What we have done in this work is so cool and awesome.
What you may criticize will all be put here as ''future work''.

This section will conclude the result of our experiments. Finally we will provide some evidence to show how pre-fetching affect fingerprinting attacks. Based on our result, we are planing to suggest some defense mechanisms.




\section{Acknowledgments}
The authors appreciate Professor Stephen McCamant for telling geeky jokes in classes all the time.
This is a research project for CSCI5271, University of Minnesota.

\bibliographystyle{abbrv}
\bibliography{csci5271} 

\end{document}
