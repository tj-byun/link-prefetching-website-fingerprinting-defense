% csci5271.tex
% this is the main LaTex file
\documentclass{sig-alternate-05-2015}
\usepackage{amsfonts}
\usepackage{graphicx}
\usepackage{hyperref}
\usepackage[hyphenbreaks]{breakurl}
\usepackage{float}

\begin{document}

\title{Link Prefetching: A Defense Against Website Fingerprinting on Tor \titlenote{This report is submitted as a partial fulfillment of {\it CSCI5271: Introduction to Security} course.}}

\numberofauthors{4}
\author{
    \alignauthor
        Vaibhav Sharma\\%
        \email{vaibhav@umn.edu}
    \alignauthor
        Taejoon Byun\\
        \email{taejoon@umn.edu}
\and
    \alignauthor
        Elaheh Ghassabani\\
        \email{ghass013@umn.edu}
    \alignauthor
        Se Eun Oh\\%
        \email{seoh@umn.edu}
\and
    \affaddr{Department of Computer Science and Engineering}  \\
        \affaddr{University of Minnesota}   \\
        \affaddr{Minneapolis, MN 55454}
}

\maketitle
\begin{abstract}
We present a novel defense mechanism to protect against website fingerprinting attacks. 
Link prefetching has been used by website developers to improve user browsing experience by anticipating what web content the user is likely to browse to next and caching it in the browser in advance. 
Link prefetching causes extra upstream and downstream packets to be introduced in the network traffic for a webpage thereby altering its fingerprint. 
We evaluate the effect of link prefetching on classifier accuracy by alternating the prefetching browser setting between training and test sets. Our results strongly suggest that link prefetching can be used to alter webpage network fingerprints regardless of the feature sets being used by the classifiers. 
We then evaluate the effect of varying the number of prefetching requests and size of prefetched responses but find the Tor browser dishonors large number of prefetch requests. 
%\emph{This paper is written to get an A in the CSCI5271 course.
%PERIOD.A++ actually}

%\emph{This content will be edited later.} We plan to explore the area of website fingerprinting in anonymization networks starting with the paper Website Fingerprinting in Onion Routing Based Anonymization Networks.
\end{abstract}

% We no longer use \terms command
%\terms{Theory}

\keywords{Website fingerprinting, anonymity, encrypted traffic, Tor}

% section 1
% TJ: let's write it later.
\section{Introduction}
% sections/intro.tex
Anonymizing networks are privacy technologies that provide a mancinism to anonymize internet communications so as to protect users from network eavesdroppers.
Although such systems are able to hide the communication (including both routing information and content), an attacker is still able to obtain different information by analyzing the network traffic.
Network analysis can provide very rich information about message length, timing, and frequency by which an attacker can easily identify the communicating parties, and therefore bypass an anonymizing system.
This problem is known as Website Fingerprinting (WF) attack, where an adversary attempts to recognize the encrypted traffic patterns of specific web pages without using any other information \cite{juarez14, murdoch2005low}.

% we should say, there is lots of work trying to prevent WF attack. And, in this project, we studied most of the state of the art works, and tries to provide a novel defend mechanism [or something similar]

% we want to add a summary about we have done

The remaining sections of this report is organized as follows: Section~\ref{sec2} provides a brief background about Tor anonymity network and website fingerprinting and Section~\ref{sec3} mentions related works on website fingerprinting attacks, defenses and criticisms.
We suggest link prefetching as a defense mechanism in Section~\ref{sec4} followed by experimental evaluation (Section~\ref{sec5})
Section~\ref{sec6} discusses about the feasibility of the proposed defense based on experimental result, and finally conclude our report on Section~\ref{sec7}.



% section 2: (Background)
\section{Website Fingerprinting on Tor}
\label{sec2}
% sections/fingerprinting.tex
% 2. Website Fingerprinting on Tor
 
This section explains the background about Tor, the anonymity network, and the website fingerprinting attack which can neutralize the anonymity that Tor provides.

\subsection{Tor}

The Onion Router (Tor) is one of the most popular anonymity networks, which distributes user's communications over several places on the Internet.
The distribution helps user's final destination not to be linked to a single point.
In other words, packets take a random path through several relays, instead of taking a direct route.
These layers cover the user path such that it will not be possible to tell where the data comes from, or where it is going.
Such a pathway in Tor is called a private network pathway.
The Tor client application is responsible of creating this private pathway by incrementally building a circuit of encrypted connections.
The circuit is constructed hop by hop, which means each individual relay is only aware of the previous and next hop.
That is to say, individual relays never can identify the complete path \cite{TorPage, dingledine2004tor}

If an attacker can see both ends of the private pathway, Tor will fail.
For example, suppose the attacker watches the Tor relay to which the user enters the network, and also watches the website he visit; in this case, the attacker can correlate volume and timing information on the two sides.
To cope with this problem, Tor came up with the idea of {\it entry guards}, where each Tor client randomly selects a few relays as entry points for her first hop.
If those relays are not observed by an attacker, the user is secure.
But, needless to say, there is always a probability of losing anonymity~\cite{TorPage}.

\subsection{Website Fingerprinting}

Tor does not provide protection against all anonymity problems. In addition WF attack is still able to bypass its privacy mechanism. In the case of Tor, WF attack would take place between the user and the Guard node, or at the Guard node itself. In general, attack models can be divided into two categories: "closed world" and "open world" scenarios.  This section describes these attack models.

In the closed world model, the classifier can successfully recognize only web pages that it has already been trained on. Therefore, in this case, we deal with a small set of censored web pages. The open world scenario is when the ability of the adversary can recognize censored pages that might have not seen before \cite{TorBlog}.


Figure~\ref{fig:attack} illustrates the attack model of website fingerprinting.
% explain about the attack model, cite the "critical evaluation" paper

\begin{figure}[h]
\includegraphics[width=0.7\columnwidth]{figures/attack_model.png}
\centering
\caption{Website Fingerprinting Attack Model~\cite{juarez14}}
\label{fig:attack}
\end{figure}



% section 3
\section{Related Work}
\label{sec3}
% sections/related.tex
% 3. Related Works

There are a decent number of works on bypassing anonymizing systems, especially Tor, on which this section provides an overview.

\subsection{Fingerprinting Attacks}
The idea of extracting information about content of encrypted SSL packets dates back to 1996 \cite{wagner96}. Later in 2002, Hintz presented an attack on an encrypting web proxy calling it website fingerprinting. The attack successfully exploited a traffic analysis-based vulnerability to detect which website a particular user is surfing \cite{hintz2003}. Since then, a lot of research has been conducted in this area trying either to propose a more realistic attack, or a more effective defense mechanism. 

One interesting study has been done in \cite{herrmann2009}, where authors applied WF attack on different anonymizing techniques, including OpenSSH, OpenVPN, Stunnel, Tor. They tried to attack 775 different websites. The result presents a low detection rate in the closed world attack model (only a 3\% success rate for 775 pages). They used Multinomial Naive Bayes classifier focusing only on packet sizes. Later on, this attack has been improved with around 90\% accuracy for 100 wbpages in \cite{wang2013improved, cai2012touching}. In \cite{cai2012touching}, authors could defeat ad hoc defense mechanisms describing a new defense scheme that provides provable security properties. Penchenko et al. \cite{panchenko11} were also among the first to report website fingerprinting attacks with reasonable accuracy on Tor. They provided a sufficient understanding of the feature set and classification framework required for this attack. In fact, they carried out a comprehensive study of WF on Tor, which considers both closed world and open world attack models.  

Generally speaking, attacks on anonymizing networks took a variety of approaches; some of the attacks tries to discover the identity of the anonymous user, others focus on uncovering the private pathway, and others attempt to identify the servers users interact with  \cite{cai2012touching}.

\subsection{Defense Mechanisms}
Usually, defense mechanisms are developed at the IP/TCP level by changing the pattern of traffic. For example, they split packets into multiple packets, or insert fake packets into the traffic, or involve padding packets. A study performed in \cite{fu2003} shows that transmission at random intervals could be a defense against analysis-based attacks. Another defense technique proposed in \cite{wright2009} uses some tricks to change a traffic pattern in a way that it looks like another pattern. However, this method does not split or disordered packets. Therefore, attacks that work without packet size information can easily defeat this mechanism. In \cite{luo2011}, Luo et al. proposed a collection of tricks at the HTTP/TCP level as a defense mechanism against some analysis attacks. For example, they changed window sizes and order of packets in the TCP stream. Also, at the HTTP level, they tried to insert some
extra data into HTTP GET headers, generates some   

\subsection{Criticism}
% cite some papers which criticize the feasibility of the fingerprinting attack
% and provide some reasons why we think fingerprinting is still a serious
% threat that we should be prepared for.



% section 4
\section{Link Prefetching as a Defense}
\label{sec4}
% sections/prefetching.tex
% TJ WILL WRITE THIS SECTION !!!

%This section explains the concept of link prefetching and discusses a possible usage of it as a defense mechanism against fingerprinting attacks on Tor.
A classifier used in a website fingerprinting attack can distinguish the differnce between two classes of packets when the difference is consisntent between them~\cite{Cai:2014kjb}. % "A systematic approach to ...", 2014
Thus, intuitively, a classifier would work the best if the difference in the same class is minimal and the difference among heterogeneous classes is high at the same time.
However, when variance among fingerprints in the same class is high, a classifier would not be able to characterize it clearly or at least the detection rate using the classifier would be low.
This is the basic idea of using prefetching as a defense mechanism; our conjecture is that we can use prefetching to manipulate the number and size of incoming and outgoing packets in order to increase the variance among packets in the same class.

This section explains the concept of link prefetching, discusses its effect on website fingerprints and argue its possible usage as a defense mechanism.

\subsection{Link Prefetching}

% introduction about link prefetching
Link prefetching is a HTML syntax that gives the web browser hints about which page the user is most likely to visit in a near future~\cite{fisher2003, fisher2004link}. 
The pages and resources to pre-fetch are specified in the web page so that the web browser can silently load them (or pre-fetch them) after an idle time.
Since the pre-fetch happens only after the page is fully loaded, it does not sacrifice the loading time of the requested web page.
Moreover, it can save the loading time for the pre-fetched pages and thus improve the user experience by caching the {\it future} contents.
It was first suggested by Mozilla Foundation in 2003 and supported by most modern browsers nowadays.

\iffalse %commented out
Once the web contents provider is reasonably certain about which links the users are most likely to visit next, it can improve the user experience by saving the loading time. 
Particularly, it is most effective if the content provider may be reasonably certain which links users are going to visit next.
Having been first suggested by Mozilla Foundation, it is adopted by most modern browsers nowadays.
Today's web browsers makes use of a specific syntax called \emph{pre-fetching}, which was proposed as a draft standard by Mozilla.
Using pre-fetching, browser can predicts documents likely to be visited by the user in the near future.
Therefore, based on the hint provided by pre-fetching a browser is able to fetch those documents a head of time.
In fact, it is the web page that provides a set of pre-fetching hints for the browser.
Then, loading the page and passing an idle time, the browser starts to pre-fetch and cache specified documents.
Needless to say, this mechanism improves efficiency.
\fi
% Insert a figure of "prefetch-network" which shows the network traffic.
% explain about how prefetching actually works 

The resources to prefetch can be simply specified in {\it HTML} using a {\tt link} tag~\cite{nottingham2010}.
For example, a {\tt link} tag {\tt <link rel="p\-refetch" href="/page2.html">} tells the browser to pre-fetch a HTML file named {\tt page2.html}.
Resources other than a {\it HTML} web page can also be pre-fetched similarly using the same syntax.
There are some variations for different types of prefetching called DNS prefetching also, (specified as {\tt <link rel="dns-prefetch" ..>}) which is supported by {\it Mozilla Firefox} and {\it Google Chrome}.
Another form of expression {\tt <link rel="prerender" ..>} also does the same job as {\tt prefetch} in {\it Google Chrome} and {\it Microsoft Internet Explorer}.

Figure~\ref{fig:network} illustrates how prefetching actually works in a browser ({\it Google Chrome}).
This page is set up arbitrarily by the authors to demonstrate link prefetching, and contains a link pre-fetch tag that specifies a big image (the image on the left side labeled as {\it prefetch}).
When the prefetching is off, this image shall be requested only when a user hovers his mouse cursor on it.
However, it can be seen on the network timeline (on the right bottom) that the image is pre-fetched right after loading the page, not when the user actually requested it.
This is indicated by a long blue bar on the second row for the file named ``{\it Very-high...}'', and it is long because the size of the file is relatively big ({\it 3.5 MB}) that it took a longer time to download.
Please also note that the time it took for loading the image when the user actually requested (by hovering his mouse on it) was very short, because the image had already been pre-fetched that the browser merely loaded it from the cache (as shown in the {\it size} column on the fourth row).

\begin{figure*}[h]
\includegraphics[width=\textwidth]{figures/prefetch-network-edited.png}
\centering
\caption{Network timeline showing pre-fetch}
\label{fig:network}
\end{figure*}


\subsection{An Example of Packet Sequences}

\begin{figure}[H]
\includegraphics[width=0.95\columnwidth]{figures/prefetch.png}
\centering
\caption{A website fingerprint for pre-fetch on/off cases}
\label{fig:prefetch}
\end{figure}

Figure~\ref{fig:prefetch} illustrates a website fingerprint in terms of inter-packet timing, where the {\it x-axis} corresponds to time in seconds and the {\it y-axis} represents the number of packets captured during a certain time interval ({\it 500 ms}).
The solid line depicts the packets captured while prefetching was enabled, and the dashed line depicts when the pre-fetch was disabled in the browser settings.
Both the cases are captured for the same web page, which is the first page of {\tt wired.com}.
However, the number of incoming and outgoing packets combined are different for each case (4055 for the {\it off} case while 4218 for the {\it on} case), because the prefetch-on case obviously requests more resources for caching purpose. 
The elapsed time for loading the whole page was also different, but this variance is mainly due to the difference in network condition as shown in the figure that the packets for the {\it on} case are more scattered throughout time compared to the pre-fetch-off case.
When we ignore the speed difference, we can see that the four peaks of the two lines are roughly the same that if we capture the packets multiple times for the same website, we would be able to characterize how the {\it fingerprint} of a specific website looks like.
Please note also that inter-packet timing is only one of many features for characterizing website fingerprint.


\subsection{Fingerprint Features}
\label{sec:features}

Since all the traffic on Tor is encrypted, fingerprinting attacks blindly analyze a sequence of packets without knowing its contents, and extract {\it features} which characterizes a fingerprint.
The attacker then trains his/her classifier on multiple packet sequences from a set of website of interest, for a set of features that attacker determines to use.
In this subsection, we summarize the definition of the four most popular features defined by Cai et al.~\cite{Cai:2014kjb} to discuss the effect of prefetching on this features in the following subsection.

{\bf Packet Sequence}:
A packet sequence $P$ can be written as $P = \langle(t_1, l_1), (t_2, l_2), ..., (t_n, l_n)\rangle$ ~\cite{Cai:2014kjb}, where $t_i$ is the interpacket time between packets $i$ and $i-1$, and $l_i$ is the byte length of the $i$-th packet.
We write $P_t$ and $P_l$ as the sequences of only the interpacket times and lengths, respectively.

{\bf Unique Packet Lengths}: 
Unique packet length is differnet when there exists a unique packet length $L$ in a sequence of packet lengths $P_l$ such that it belongs to a sequence while it does not to the other. In other words:
\begin{equation}
(\exists L \in P_l | L \notin P'_l ) \vee (\exists L \in P'_l | L \notin P_l)
\end{equation}

{\bf Packet Length Frequency}:
When $n_L(P_l)$ is the number of times packet length $L$ appears in $P_l$,
\begin{equation}
\exists L|n_L(P_l) \neq n_L(P'_l) \wedge n_L(P_l)>0 \wedge n_L(P'_l) >0
\end{equation}
In other words, $P$ and $P'$ have different packet length frequencies iff there exists a packet with length $L$ that appears different times in the two sequences.

{\bf Packet Ordering}:
When $M_l$ is the multiset of packet lengths in $P_l$ without ordering, two packet sequences $P$ and $P'$ have different packet ordering iff:
\begin{equation}
M_{ l }=M'_{ l }\wedge P_{ l }\neq P'_{ l }
\end{equation}

{\bf Interpacket Timing}:
Interpacket timing is different if the packets of the same index from the two different sequences $P$ and $P'$ have different interpacket timing:
\begin{equation}
\exists i, 1 \le i \le \mathit{min}(|P|, |P'|) : (P_t)_i \neq (P'_t)_i
\end{equation}

Based on these features, Cai et al.~\cite{Cai:2014kjb} claimed that if $P \neq P'$, then one of the four features shall differ.


\subsection{The Effect of Prefetching on the Features}

Based on the features we summarized in Section~\ref{sec:features}, this subsection explain how prefetching can affect the four features.
We will compare the two packet sequences $P$ and $P'$ for the same webpage, where $P$ is the packet sequence without prefetching and $P'$ is the packet sequence where prefetching was enabled (and where there is more than one pre-fetch request).

{\bf Packet Sequence}:
By the definition of link prefetching~\cite{fisher2004link}, pre-fetch is requested only after a webpage is fully loaded and after an idle time.
Thus, for a packet sequence $P$ where prefetching was disabled, another packet sequence $P'$ with prefetching enabled would not differ except the prefix of $P'$.
In other words, $P'$ is a concatenation of two sequences $P$ and $Q$ such that $P' = P \Vert Q$, where $Q$ is the incoming and outgoing packets for pre-fetch request and download such that:
\begin{equation}
Q = \langle(t_{n+1}, l_{n+1}), (t_{n+2}, l_{n+2}), ..., (t_{n+m}, l_{n+m})\rangle, m > 1
\end{equation}
%We assume a prefetching packet sequence $P'$ (a packet sequence when link prefetching was enabled and more than one prefetching request is made) simply as an extension of a sequence $P$ where prefetching was disabled.
%This is a valid assumption because by the definition of link prefetching~\cite{fisher2014link}, all the request packets and the download packets for prefetching will follow only after the page is fully loaded.
%When the number of packets when prefetching was off is $n$ and the number of prefetching packets are $m$, the packet sequence can be written as $Q = \langle(t_1, l_1), (t_2, l_2), ..., (t_n, l_n), (t_{n+1}, l_{n+1}), ..., (t_m, l_m)\rangle$.

{\bf Unique Packet Lengths}: 
In order for the {\it unique packet lengths} to be different between $P$ and $P'$, it is sufficient to have one packet with a different length in $P'$ that do not appear in $P$, or vice versa.
When there are tailing packets $Q$ for $P'$, the likelihood is high that there is one or more packets in $Q$ that do not exist in $P$.
Although in some cases the multiset of packet lengths $M'_l$ can be a subset of $M_l$, it is not likely considering that the prefetching resources are different ones from the resources that are already fetched -- the designer of a website would not want to pre-fetch the same resource that's already been downloaded.
Thus, unique packet lengths will be different between $P$ and $P'$.

{\bf Packet Length Frequency}:
This feature differs when there exists a packet length $L$ such that the frequencies differ between $P$ and $P'$, while appearing at least once in both $P$ and $P'$.
This is highly likely to be different considering that most the packets are transmitted with the size of MTU (maximum transmission unit), which is typically 1500 bytes.
When the requested resource is bigger than 1500 bytes, which is true in most cases, the number of packts with length 1500 will be larger in $P'$ than in $P$ ($n_{1500}(P') > n_{1500}(P)$) and thus yield false for this feature.

{\bf Packet Ordering}:
$P$ and $P'$ obviously do not have the same packet ordering because $P_l \neq (P \Vert Q)_l$.

{\bf Interpacket Timing}:
This feature would not be affected because of the clause $\mathit{min}(|P|,|P'|)$ would always yield $|P|$, and $(P_t)_i$ where $i$ is between $1$ and $|P|$ would not be affected by $Q$.

In summary, at least one out of features is affected by link prefetching.
Furthermore, if our assumptions about the prefetching packets are valid, at most three features are affected by link prefetching, which is sufficient for a classifier to put $P'$ in a different class to $P$.

We show the effect of prefetching in various settings throughout a series of experiments.



% section 5
\section{Experimental Evaluation}
\label{sec5}
% sections/experiment.tex
\begin{table}[]
\centering
\caption{Experiment design to answer {\it RQ1}}
\label{table:prefetch}
\begin{tabular}{lllll}
\cline{1-3}
\multicolumn{1}{|l|}{victim \textbackslash attacker} & \multicolumn{1}{l|}{prefetch on} & \multicolumn{1}{l|}{prefetch off} &  &  \\ \cline{1-3}
\multicolumn{1}{|l|}{prefetch on}                    & \multicolumn{1}{l|}{(1)}         & \multicolumn{1}{l|}{(2)}          &  &  \\ \cline{1-3}
\multicolumn{1}{|l|}{prefetch off}                   & \multicolumn{1}{l|}{(3)}         & \multicolumn{1}{l|}{(4)}          &  &  \\ \cline{1-3}
                                                     &                                  &                                   &  & 
\end{tabular}                  
\end{table}

\begin{enumerate}
\item
We speculate that prefetching itself might provide extra defense because of the extra packets.
It can also be the case however, prefetching websites are more vulnerable to fingerprinting because of the extra prefetch packets that shows distinct prefix.
\item
This case is unlikely since prefetching is on by default. We assume that victims will more likely be using Tor under the default setting.
\item
This case is what we are most curious about, whether a victim can confuse an attacker by simply turning prefetching setting off of his browser.
\item
This case simulates a situation where a victim is loading any other websites that does not prefetch any resource.
This can be used as a comparison case.
\end{enumerate}


% section 6
\section{Discussion}
\label{sec6}
% sections/discussion.tex
The performance of link prefetching as a defense mechanism depends on the values used by the webpage for the number of prefetching requests made and the total size of prefetched responses. 
A large number of outgoing requests and incoming responses would make the network prefetching traffic indistinguishable from the normal traffic required to display the webpage.
If a webpage \textit{A} is to make its own network traffic fingerprint similar to another webpage  \textit{B}'s network traffic fingerprint, assuming that webpage \textit{A} currently has a smaller fingerprint than webpage \textit{B}, it should create prefetching network traffic to compensate for the difference in the network traffic fingerprints of \textit{A} and \textit{B}. 
This can further be generalized to the open world model where every webpage can trigger a set of prefetching requests and responses which change its fingerprint to be more similar to a different webpage which has a larger fingerprint. 
This can work for fingerprints which use coarse-grained features as well as fine-grained features.
e.g. a webpage \textit{A} that finds another webpage \textit{B} that creates \textit{K} more outgoing packets resulting in an increase in the total incoming packet size of \textit{S} can simply issue \textit{K} prefetching requests for resources that have a total size of \textit{S}. 
It should however be noted that webpage \textit{A} would have to make requests for resources that are unlikely to be currently cached by the browser.\\
Some attacks may also choose to look at only the first 50 or 100 packets that show up in a packet capture for a given webpage and try to develop a fingerprint using only those packets. 
While the question of the number of packets to be used to create a fingerprinting is definitely interesting, it also needs to be further investigated how many prefetching request and response packets appear in such packet captures. 


% section 7
\section{Conclusions}
\label{sec7}
% sections/conclusion.tex

What we have done in this work is so cool and awesome.
What you may criticize will all be put here as ''future work''.

This section will conclude the result of our experiments. Finally we will provide some evidence to show how pre-fetching affect fingerprinting attacks. Based on our result, we are planing to suggest some defense mechanisms.




\section{Acknowledgments}
This report is submitted as a partial requirement for the CSCI5271 research project.
The authors acknowledge the guidance of Professor Stephen McCamant in the University of Minnesota throughout the course of this research project.

\bibliographystyle{abbrv}
\bibliography{csci5271}

\end{document}
