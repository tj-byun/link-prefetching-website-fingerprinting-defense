% sections/fingerprinting.tex
% 2. Website Fingerprinting on Tor
 
This section explains the background about Tor, the anonymity network, and the website fingerprinting attack which can neutralize the anonymity that Tor provides.

\subsection{Tor}

The Onion Router (Tor) is one of the most popular anonymity networks, which distributes user's communications over several places on the Internet.
The distribution helps user's final destination not to be linked to a single point.
In other words, packets take a random path through several relays, instead of taking a direct route.
These layers cover the user path such that it will not be possible to tell where the data comes from, or where it is going.
Such a pathway in Tor is called a private network pathway.
The Tor client application is responsible of creating this private pathway by incrementally building a circuit of encrypted connections.
The circuit is constructed hop by hop, which means each individual relay is only aware of the previous and next hop.
That is to say, individual relays never can identify the complete path \cite{TorPage, dingledine2004tor}

If an attacker can see both ends of the private pathway, Tor will fail.
For example, suppose the attacker watches the Tor relay to which the user enters the network, and also watches the website he visit; in this case, the attacker can correlate volume and timing information on the two sides.
To cope with this problem, Tor came up with the idea of {\it entry guards}, where each Tor client randomly selects a few relays as entry points for her first hop.
If those relays are not observed by an attacker, the user is secure.
But, needless to say, there is always a probability of losing anonymity~\cite{TorPage}.

\subsection{Website Fingerprinting}

Tor does not provide protection against all anonymity problems. In addition WF attack is still able to bypass its privacy mechanism. In the case of Tor, WF attack would take place between the user and the Guard node, or at the Guard node itself. In general, attack models can be divided into two categories: "closed world" and "open world" scenarios.  This section describes these attack models.

In the closed world model, the classifier can successfully recognize only web pages that it has already been trained on. Therefore, in this case, we deal with a small set of censored web pages. The open world scenario is when the ability of the adversary can recognize censored pages that might have not seen before \cite{TorBlog}.


Figure~\ref{fig:attack} illustrates the attack model of website fingerprinting.
% explain about the attack model, cite the "critical evaluation" paper

\begin{figure}[h]
\includegraphics[width=0.7\columnwidth]{figures/attack_model.png}
\centering
\caption{Website Fingerprinting Attack Model~\cite{juarez14}}
\label{fig:attack}
\end{figure}

